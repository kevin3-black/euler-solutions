\documentclass[11pt]{article}
\usepackage[utf8]{inputenc}
\usepackage{graphicx}
\usepackage{amsmath}
\usepackage{amsfonts}
\usepackage{hyperref}
\usepackage{listings}
\usepackage{courier}
\usepackage[letterpaper, portrait, margin=1in]{geometry}
\graphicspath{/home/black/CPS/euler}

\lstset{
	basicstyle=\footnotesize\ttfamily,
	frame=single,
	breaklines=true
}

\title{\vspace{-1.5cm}Project Eueler 512}
\author{Kevin Black}

\begin{document}

\maketitle

\section{Problem Statement}

Let $\phi(n)$ be Euler's totient function.\\

\noindent
Let $f(n) = \left(\sum^n_{i=1}\phi(n^i)\right) \mod (n + 1)$.\\

\noindent
Let $g(n) = \sum^n_{i=1}f(i)$.\\

\noindent
$g(100) = 2007$.\\

\noindent
Find $g(5 \times 10^8)$.\\

\section{Simplification}

\begin{align*}
f(n) &= \phi(n) + \phi(n^2) + \phi(n^3) + ... + \phi(n^n) \mod (n+1)\\
&= \phi(1 \cdot n) + \phi(n \cdot n) + \phi (n^2 \cdot n) + ... + \phi(n^{n-1} \cdot n) \mod (n+1)\\
&= 1 \cdot \phi(n) + n \cdot \phi(n) + n^2 \cdot \phi(n) + ... + n^{n-1} \cdot \phi(n) \mod (n+1)\\
&= \phi(n)\left(1 + n + n^2 + n^3 + ... + n^{n-1}\right) \mod (n+1)\\
&= \left(\phi(n) \mod (n+1)\right)\left((1 + n + n^2 + n^3 + ... + n^{n-1}) \mod (n+1)\right) \mod (n+1)
\end{align*}

Ignoring the $\phi(n)$ for now, and taking the right-hand side of the multiplication:
\begin{align*}
1 + n + n^2 + ... + n^{n-1} &\equiv (1 + n + n^2 + n^3 + ... + n^{n-1}) - (n+1) &\mod (n+1)\\
&\equiv n^2 + n^3 + ... + n^{n-1} &\mod (n+1) \\
&\equiv n^2 (1 + n + n^2 + n^3 + ... + n^{n-3}) &\mod (n+1)\\
&\equiv \left(n^2 \mod (n+1)\right)\left((1 + n + n^2 + n^3 + ... + n^{n-3}) \mod (n+1)\right) &\mod (n+1)\\
&\equiv 1 \cdot \left((1 + n + n^2 + n^3 + ... + n^{n-3}) \mod (n+1)\right) &\mod (n+1)\\
&\equiv 1 + n + n^2 + n^3 + ... + n^{n-3} &\mod (n+1)
\end{align*}

The second-to-last step, where the $(n^2 \mod (n+1)$ is eliminated, is due to the fact that $n^2 \equiv (-1)^2 \equiv 1 \mod (n+1)$.

This process of eliminating $(1+n)$ and factoring out $n^2$ can be repeated, shrinking the series until there are no more terms left. If $n$ is even, then the series will eventually become $(1 + n) \mod (n+1)$, and subtracting out the final $(n + 1)$ gives 0. If $n$ is odd, then the series will eventually become $(1 + n + n^2) \mod (n+1)$, and subtracting out the final $(n+1)$ gives $n^2 \mod (n+1)$, which is 1. Therefore, the final result is:

\begin{equation*}
1 + n + n^2 + n^3 + ... + n^{n-1} \mod (n+1) = \begin{cases}
0 &\text{$n$ is even}\\
1 &\text{$n$ is odd}
\end{cases}
\end{equation*}

Plugging that back into the original equation, it gives:

\begin{equation*}
f(n) = \begin{cases}
0 &\text{$n$ is even}\\
\phi(n) &\text{$n$ is odd}
\end{cases}
\end{equation*}

The ``$\text{mod } (n+1)$'' can be dropped because $\phi(n)$ will never be greater than $n - 1$.\\

This all means that $g(n)$ can be simplified to $\phi(1) + 0 + \phi(3) + 0 + \phi(5) + 0 + ... + \phi(n)$, or the sum of all odd totients between 1 and $n$, inclusive. This is still not trivial to compute for $n = 5 \cdot 10^8$, which the code below does.

\section{Code}
\lstinputlisting[language=Python]{512.py}

\section{Rationale}

The function \verb!computeOddTotients(n)! returns a list of $\phi(i)$ for all odd $i$ between 1 and $n$, inclusive. It does so using a similar method to the Sieve of Eratosthenes to find all of the prime factors of each $i$, and uses Eueler's product formula to compute the totient of $i$, which is: 

\begin{equation*}
\displaystyle
\phi(i) = i\prod_{p|i}(1 - \frac{1}{p})
\end{equation*}

A list of length $5 \cdot 10^8$ is extremely large and was very slow with the amount of memory I had available. Fortunately, we only care about odd totients, so we only technically need half as many elements, which is far more manageable. I accomplish this by creating a list \verb!phi! encoded such that \verb!phi[i]! $= \phi(2i+1)$, covering odd numbers only. I initialize each element to the number it is supposed to represent, $2i + 1$; this indicates that the number has not been ``visited'' yet.

The Sieve-of-Eratosthenes-like algorithm then iterates through the entire list \verb!phi!. If \verb!phi[i]! is still equal to the odd number it's supposed to represent ($2i+1$), then it has not been  ``visited'' yet, and $2i + 1$ (called $p$ from now on) is prime. That means \verb!phi[i]! can be directly computed as $p-1$. The algorithm then ``visits'' all of the multiples of the prime $p$, and changes them, marking them as not prime.

For each $j$ that is a multiple of $p$ that the algorithm visits, $p$ is a prime factor of $j$, and thus Eueler's product formula can be used with $p$ and $j$. Each $j$ starts out with the value $j$ (itself), accounting for the term outside of the product in Eueler's product formula. Then, each time $j$ is visited by a prime factor $p|j$, it is multiplied by $(1-\frac{1}{p})$. (Note: in the actual code, $j$ is distributed over $(1-\frac{1}{p})$ in order to avoid floating point division). Once the algorithm completes, every $j$ will have been visited by every $p|j$, meaning every term in Euler's product formula will have been accounted for, and \verb!phi! will hold all of the correct totients.

\end{document}